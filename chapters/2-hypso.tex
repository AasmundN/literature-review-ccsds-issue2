\chapter{HYPerspectral Smallsat for Ocean observation (HYPSO)}

The HYPerspectral Smallsat for Ocean observation (HYPSO) is a project at the Norwegian University of Science and Technology (NTNU) focusing on hyperspectral imagery for ocean observation. The HYPSO-1 and HYPSO-2 satellites, launched in 2022 and 2024 respectively, both carry hyperspectral imagers capable of capturing data in the spectral range within wavelengths 400-800$nm$ \cite{grotteOceanColorHyperspectral2022}. This spectral range allows monitoring of algal blooms, a phenomenon giving important clues as to the health of marine environments and ecosystems \cite{blondeau-patissierReviewOceanColor2014}.

The sensors onboard the HYPSO satellites produce images with 120 spectral bands at spatial sizes of 598 lines and 1092 samples for HYPSO-2, and 684 by 956 for HYPSO-1 \cite{grotteOceanColorHyperspectral2022}. This, combined with the addition of meta data such as georeferencing and anomaly detection, produce large amounts of data to be stored, processed and downlinked. A typical raw HYPSO-1 image has a size of 160\textit{MB}, and the, somewhat conservative, downlink capability during a single orbital revolution is 75\textit{MB}. In order to achieve acceptable data latency it is evident that one must utilise image compression before downlinkning the data. Small satellites often operate at strict power budgets, and with limited hardware resources. With many onboard systems to manage, processor time is also limited. Commercially available Field Programmable Gate Arrays (FPGAs) tailored for space applications exist. By combining the flexibility of reconfiguration and speed of a custom hardware design, these are the ideal choice for implementation of HSI compression.
