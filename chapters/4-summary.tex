\chapter{Summary}

With the advent of small satellite technology, such as the HYPSO project at NTNU, hyperspectral imagery has become readily available. These image sensors produce large amounts of data which needs to be downlinked to earth using channels with limited bandwidth. This prompts the need for onboard compression using algorithms such as CCSDS-123.0-B-2. Hardware implementations for FPGA of the algorithm must balance hardware area footprint, compression performance and data throughput. Three main approaches have been proposed to alleviate the spectral and spatial data dependencies that make these requirements hard to fulfill. \citeauthor{basconesRealTimeFPGAImplementation2022} suggest a careful rearranging of the input samples, giving data dependencies time to resolve and allowing deep pipelining. \citeauthor{chatziantoniouScalableDataRateNearLossless2022a} uses pre-quantization, at the loss of some compression performance, to remove the in-loop quantization and sample representatives. Both of these approaches reach a throughput of 285$Msamples/s$ using a single core, with the possibility of reaching higher performance with image segmentation and multicore designs. A better approach for hardware resource constrained systems is the mathematical optimization approach combined with reduced weight updating frequency, proposed by \citeauthor{sanchezReducingDataDependencies2022} and \citeauthor{jiaRemovalFeedbackLoop2025} respectively. The latter presents a predictor with a throughput of 384$Msamples\s$ with a small hardware footprint of only 3995 LUTs and 5050 FFs.
